%%%%%%%%%%%%%%%%%%%%%%%%%%%%%%%%%%%%%%%%%%%%%%%%%%%%%%%%%%%%%%%%%%%%%%
% LaTeX Template: Curriculum Vitae
%
% Source: http://www.howtotex.com/
% Feel free to distribute this template, but please keep the
% referal to HowToTeX.com.
% Date: July 2011
% 
%%%%%%%%%%%%%%%%%%%%%%%%%%%%%%%%%%%%%%%%%%%%%%%%%%%%%%%%%%%%%%%%%%%%%%
% How to use writeLaTeX: 
%
% You edit the source code here on the left, and the preview on the
% right shows you the result within a few seconds.
%
% Bookmark this page and share the URL with your co-authors. They can
% edit at the same time!
%
% You can upload figures, bibliographies, custom classes and
% styles using the files menu.
%
% If you're new to LaTeX, the wikibook is a great place to start:
% http://en.wikibooks.org/wiki/LaTeX
%
%%%%%%%%%%%%%%%%%%%%%%%%%%%%%%%%%%%%%%%%%%%%%%%%%%%%%%%%%%%%%%%%%%%%%%
\documentclass[paper=a4,fontsize=11pt]{scrartcl} % KOMA-article class
							
\usepackage[english]{babel}
\usepackage[utf8x]{inputenc}
\usepackage[protrusion=true,expansion=true]{microtype}
\usepackage{amsmath,amsfonts,amsthm}     % Math packages
\usepackage{graphicx}                    % Enable pdflatex
\usepackage[svgnames]{xcolor}            % Colors by their 'svgnames'
\usepackage{geometry}
	\textheight=700px                    % Saving trees ;-)
\usepackage{url}

\frenchspacing              % Better looking spacings after periods
\pagestyle{empty}           % No pagenumbers/headers/footers

%%% Custom sectioning (sectsty package)
%%% ------------------------------------------------------------
\usepackage{sectsty}

\sectionfont{%			            % Change font of \section command
	\usefont{OT1}{phv}{b}{n}%		% bch-b-n: CharterBT-Bold font
	\sectionrule{0pt}{0pt}{-5pt}{3pt}}

%%% Macros
%%% ------------------------------------------------------------
\newlength{\spacebox}
\settowidth{\spacebox}{8888888888}			% Box to align text
\newcommand{\sepspace}{\vspace*{1em}}		% Vertical space macro

\newcommand{\MyName}[1]{ % Name
		\Huge \usefont{OT1}{phv}{b}{n} \hfill #1
		\par \normalsize \normalfont}
		
\newcommand{\MySlogan}[1]{ % Slogan (optional)
		\large \usefont{OT1}{phv}{m}{n}\hfill \textit{#1}
		\par \normalsize \normalfont}

\newcommand{\NewPart}[1]{\section*{\uppercase{#1}}}

\newcommand{\PersonalEntry}[2]{
		\noindent\hangindent=2em\hangafter=0 % Indentation
		\parbox{\spacebox}{        % Box to align text
		\textit{#1}}		       % Entry name (birth, address, etc.)
		\hspace{1.5em} #2 \par}    % Entry value

\newcommand{\SkillsEntry}[2]{      % Same as \PersonalEntry
		\noindent\hangindent=2em\hangafter=0 % Indentation
		\parbox{\spacebox}{        % Box to align text
		\textit{#1}}			   % Entry name (birth, address, etc.)
		\hspace{1.5em} #2 \par}    % Entry value	
		
\newcommand{\EducationEntry}[4]{
		\noindent \textbf{#1} \hfill      % School
        #2 \par
%		\fbox{#2} \par                    % Duration
		\noindent \textit{#3} \par        % Study
		\noindent\hangindent=2em\hangafter=0 \small #4 % Description
		\normalsize \par}

\newcommand{\WorkEntry}[4]{				  % Same as \EducationEntry
		\noindent \textbf{#1} \hfill      % Company
        #2 \par
%		\fbox{#2} \par                    % Duration
		\noindent \textit{#3} \par        % Jobname
		\noindent\hangindent=2em\hangafter=0 \small #4 % Description
		\normalsize \par}

%%% Begin Document
%%% ------------------------------------------------------------
\begin{document}
% you can upload a photo and include it here...
%\begin{wrapfigure}{l}{0.5\textwidth}
%	\vspace*{-2em}
%		\includegraphics[width=0.15\textwidth]{photo}
%\end{wrapfigure}

\MyName{Kenneth Laskoski}
%\MySlogan{Curriculum Vitae}

\sepspace

%%% Personal details
%%% ------------------------------------------------------------
\NewPart{Contato}{}

%\PersonalEntry{Birth}{6 de outubro de 1974}
\PersonalEntry{Endereço}{Rua Felipe de Oliveira, 566/404 - Petrópolis - Porto Alegre RS}
\PersonalEntry{Telefone}{(51) 99428-4677}
\PersonalEntry{Email}{\url{kennethlaskoski@gmail.com}}

%%% Education
%%% ------------------------------------------------------------
\NewPart{Formação}{}

\EducationEntry{Universidade Federal do Rio Grande do Sul}{março 2016 - presente}{Bacharelado em Física}{Graduação em andamento.}
\sepspace

%%% Work experience
%%% ------------------------------------------------------------
\NewPart{Experiência Profissional}{}

\WorkEntry{Nelogica Sistemas de Software}{novembro 2018 - outubro 2020}{Desenvolvedor Sênior}{Desenvolvimento de app iOS para negociação de títulos na Bolsa de Valores. Criação de plataforma de transmissão de dados binários, entre os servidores e os dispositivos móveis, obedecendo parâmetros rígidos de velocidade e confiabilidade. Participação na implementação de apresentação gráfica dos dados, permitindo a negociação diretamente na tela de gráficos. Integração do desenvolvimento e publicação do app utilizando ferramentas e metodologias atuais, como GitLab, Jira e Carthage.}
\sepspace
\sepspace

\WorkEntry{RZ2 Sistemas de Gestão}{abril 2017 - novembro 2018}{Desenvolvedor Sênior}{Desenvolvimento de aplicativos para iPhone e iPad com uso da linguagem Swift. O app desenvolvido implementa um sistema de questionário customizável que pode ser utilizado em diversos cenários, destacando-se controle de estoque, auxílio a logística e otimização de tarefas. A lista de clientes atendidos inclui Lojas Marisa, Camisaria Colombo, Toyota, Nissan, Renault, Pizza Hut, KFC, Habib’s, Petróleo Ipiranga.}
\sepspace
\sepspace

\WorkEntry{Lexsis Sistemas}{março 2016 - abril 2017}{Desenvolvedor Sênior}{Desenvolvimento de aplicativos para smartphones e tablets Android e iOS nas linguagens Java e Swift, respectivamente. Os aplicativos são voltados ao ramo de restaurantes.}
\sepspace
\sepspace

\WorkEntry{It’s Mobile Tecnologia e Inovação}{maio 2013 - março 2016}{Desenvolvedor}{Desenvolvimento de aplicativos para tablets iOS em Objective-C, com foco a automação da força de vendas de forma integrada com sistemas SAP, para clientes como Bettanin, Pado, Randon, Dudalina, entre outros.}

\newpage

\WorkEntry{ITM Consultoria e Negócios - Intermídia}{julho 2012 - maio 2013}{Desenvolvedor}{Desenvolvimento de aplicativos para dispositivos móveis, em particular Apple iPad, com foco na automação da força de vendas da indústria calçadista (Arezzo, Crysalis, Pegada, Piccadilly) utilizando, principalmente, Objective-C e SQLite.}
\sepspace
\sepspace

\WorkEntry{DBKi Informática}{setembro 2011 – julho 2012}{Desenvolvedor}{Desenvolvimento e manutenção de webservices integrantes do sistema de e-commerce das Lojas Renner, utilizando as tecnologias .NET (C\texttt{\#}). Desenvolvimento de aplicativos para sistemas iOS (iPhone e iPad), tendo como cliente as Lojas Renner, entre outros, com a utilização da linguagem Objective-C na plataforma Apple.}
\sepspace
\sepspace

\WorkEntry{T\&T Engenheiros Associados}{setembro 2009 – setembro 2011}{Desenvolvedor}{Desenvolvimento e manutenção da Conformance Test Suite responsável pelo teste de homologação de sistemas candidatos à certificação DASH (Desktop and mobile Architecture for System Hardware), segundo standard definido e publicado pela DMTF (Distributed Management Task Force), consórcio do qual são membros: HP, Intel, Microsoft, Dell; entre outros. A suite DASH CTS é composta de programas escritos em Java e C\texttt{++}, além de scripts de linha de comando.}
\sepspace
\sepspace

\WorkEntry{DBServer Assessoria em Sistemas de Informação}{novembro 2005 – novembro 2008}{Desenvolvedor}{Participação em projeto da Tlantic SI, empresa integrante do grupo Sonae, objetivando o desenvolvimento de sistema de automação instalado nas lojas do grupo em Portugal e outros países. Emprego de C\texttt{++}, XML, webservices, patterns, UML, nos sistemas operacionais Linux e Windows.}
\sepspace
\sepspace

\WorkEntry{CWI Informática}{fevereiro 2002 – abril 2005}{Desenvolvedor}{Integrante da equipe de manutenção de sistema responsável por gerenciar informações no cliente Agrofel, incluindo controle de vendas e financiamentos. Software escrito basicamente em C e implantado na plataforma HP9000.}
\sepspace
\sepspace

\WorkEntry{Departamento de Engenharia Química - UFRGS}{março 1998 – fevereiro 2000}{Bolsista de Apoio Técnico}{Desenvolvedor em projeto de sofware, patrocinado por Petrobras e Braskem, especializado na supervisão e avaliação de sistemas de controle da indústria química/petroquímica. Programação em C\texttt{++} utilizando o framework Qt para implementar GUI multiplataforma Windows/Linux.}
\sepspace
\sepspace

\WorkEntry{Lógica Informática}{abril 1990 – novembro 1991}{Programador}{Programação em Clipper de sistemas destinados a pequenas e médias empresas, tais como Folha de Pagamento, Contabilidade, Contas a Pagar/Receber.}

%%% Skills
%%% ------------------------------------------------------------
\NewPart{Habilidades}{}

\SkillsEntry{Línguas}{Inglês fluente}

\SkillsEntry{Software}{iOS, Swift, Scrum, DevOps, C/C\texttt{++}, \LaTeX}


%%% References
%%% ------------------------------------------------------------
%%\NewPart{References}{}
%%Available upon request
\end{document}
